\documentclass[12pt,a4paper,landscape]{article}
\usepackage[utf8]{inputenc}
\usepackage[english]{babel}
\usepackage{amsmath}
\usepackage{amsfonts}
\usepackage{amssymb}
\usepackage{makeidx}
\usepackage{graphicx}
\usepackage{breqn}
\usepackage{mathtools}
\usepackage{geometry}
\usepackage{caption}
\geometry{landscape, margin=1in}
\author{Karthik J\\M.Sc.Ed. Physics}
\title{8th Semester\\\textsc{Numerical Analysis}\\Formula Sheet\\\vspace{14pt}Based on the lectures by \\\textsc{Ajay Kumar K},\\\textsc{Regional Institute of Education Mysore}}
\date{2018-19}
%\usepackage{theorem}
\usepackage{amsthm}

\newtheorem{thm}{Theorem}
\newtheorem*{rmk}{Remark}
\usepackage{fancyhdr}
\usepackage[light,condensed,math]{kurier}
\pagestyle{fancy}
\fancyhf{}
\fancyhead[LE,RO]{Karthik J}
\fancyhead[RE,LO]{Numerical Analysis}
\fancyfoot[LE,RO]{Page \thepage}
\usepackage{xcolor}
\usepackage{graphicx}
\definecolor{titlepagecolor}{cmyk}{1,.60,0,.40}
\definecolor{namecolor}{cmyk}{1,.50,0,.10} 
\definecolor{jade}{rgb}{0.0, 0.66, 0.42}
\newcommand{\dydx}{\dfrac{dy}{dx}}
\begin{document}
	\begin{titlepage}
		%\newgeometry{left=7.5cm} %defines the geometry for the titlepage
		\pagecolor{jade}
		\noindent
		\color{white}
		\makebox[0pt][l]{\rule{1.3\textwidth}{1pt}}
		\par
		\noindent
		\textbf{\textsf{Karthik J}} \textcolor{namecolor}{\textsf{Regional Institute of Education Mysore}}
		\vfill
		\noindent
		{\huge \textsf{Numerical Analysis\\Formula Sheet}}
		\vskip\baselineskip
		\noindent
		\textsf{Based on the lectures by\\\textsc{Ajay Kumar K},\\\textsc{Regional Institute of Education Mysore}}
	\end{titlepage}
	\restoregeometry % restores the geometry
	\nopagecolor
	\maketitle
	\newpage
	\tableofcontents
	\section{Interpolation}
	\subsection{Weierstrass' Approximation Theorem}
	\begin{thm}
		Let $f$ be a continuous function defined on $[a,b]$. Then, given any $\varepsilon>0$, $\exists$ a polynomial $p(x)$, such that 
		\[ \left| f(x) - p(x) \right| < \varepsilon \]
		$\forall\ x \in [a,b]$.
	\end{thm}
	
	\subsection{Newton-Gregory Forward Difference Formula}
		Consider a set of equally-spaced finite number of data points:
		\begin{center}
			\begin{tabular}{|c|c|}
				\hline
				$x_0$& $y_0$\\
				$x_1$&	$y_1$\\
				$x_2$&	$y_2$\\
				$\vdots$&	$\vdots$\\
				$x_n$&	$y_n$\\
				\hline
			\end{tabular}
		\end{center}
		where $x_i - x_{i-1} = h\ \forall\ i = 1,2,\ldots,n$		\\
		The differences $y_1 - y_0,y_2 - y_1 \ldots, y_n - y_{n-1} $, when denoted by $\Delta y_1, \Delta y_2, \ldots, \Delta y_n$ respectively are called the \textit{first forward differences} and $\Delta$ is called the \textbf{forward difference operator}.
		\begin{equation}\label{for-op}
			\Delta y^r_i = y^{r-1}_{i+1} - y^{r-1}_{i}
		\end{equation}
		For example, $\Delta^1 y_0 = y_1 - y_0$, $\Delta^2 y_0 = \Delta^1 y_1 - \Delta^1 y_0$, $\Delta^3 y_0 = \Delta^2 y_1 - \Delta^2 y_0$  and so on.\\
		\begin{center}
			\begin{tabular}{cccccc}
				\hline
				$x$-value& $y$-value& 1st diff.& 2nd diff.& 3rd diff.& 4th diff\\ \hline
				$x_0$&	$y_0$&	&&&\\
				&& $\Delta y_0$&&&\\
				$x_0 + h$&	$y_1$	& & $\Delta^2 y_0$&&\\
				&& $\Delta y_1$& & $\Delta^3 y_0$&\\
				$x_0 + 2h$&	$y_2$	& & $\Delta^2 y_1$&& $\Delta^4 y_0$\\ 
				&& $\Delta y_2$& & $\Delta^3 y_1$&\\
				$x_0 + 3h$&	$y_3$	& & $\Delta^2 y_2$&&\\
				&& $\Delta y_3$&&&\\
				$x_0 + 4h$&	$y_4$	& & &&\\ \hline
			\end{tabular}\\
		\end{center}
		Let
		\[ y(x) = a_0 + a_1 (x - x_0) + a_2 (x-x_0)(x-x_1) + \cdots + a_n (x-x_0)(x-x_1)\ldots (x-x_{n-1}) \]
		be a polynomial that satisfies the $n+1$ tabulated data. Then, the expression becomes:
		\begin{dmath}\label{newton-for}			
		y(x) = y_0 + \dfrac{\Delta y_0}{h} (x-x_0) + \dfrac{\Delta^2 y_0}{2! h^2} (x-x_0)(x-x_1)+ \cdots + \dfrac{\Delta^n y_0}{n! h^n} (x-x_0)(x-x_1)\cdots (x-x_{n-1})
		\end{dmath}
		Now, using the substitution $\dfrac{x-x_0}{h} = u$, the equation \ref{newton-for} becomes:
		\begin{equation}\label{newton-for-u}
			\boxed{y(u) = y_0 + \Delta y_0u + \dfrac{\Delta^2 y_0}{2!} u(u-1) + \cdots + \dfrac{\Delta^n y_0}{n!} u(u-1)(u-2)\cdots (u-(n-1))}
		\end{equation}
		
		\subsection{Newton-Gregory Backward Difference Formula}
		The differences $y_1 - y_0,y_2 - y_1 \ldots, y_n - y_{n-1} $, when denoted by $\nabla y_1, \nabla y_2, \ldots, \nabla y_n$ respectively are called the \textit{first backward differences} and $\nabla$ is called the \textbf{backward difference operator}.
		\begin{equation}\label{back-op}
		\nabla y^r_i = y^{r-1}_{i+1} - y^{r-1}_{i}
		\end{equation}
		For example, $\nabla^1 y_0 = y_1 - y_0$, $\nabla^2 y_0 = \nabla^1 y_1 - \nabla^1 y_0$, $\nabla^3 y_0 = \nabla^2 y_1 - \nabla^2 y_0$  and so on.\\
		\begin{center}
			\begin{tabular}{cccccc}
				\hline
				$x$-value& $y$-value& 1st diff.& 2nd diff.& 3rd diff.& 4th diff\\ \hline
				$x_0$&	$y_0$&	&&&\\
				&& $\nabla y_0$&&&\\
				$x_0 + h$&	$y_1$	& & $\nabla^2 y_0$&&\\
				&& $\nabla y_1$& & $\nabla^3 y_0$&\\
				$x_0 + 2h$&	$y_2$	& & $\nabla^2 y_1$&& $\nabla^4 y_0$\\ 
				&& $\nabla y_2$& & $\nabla^3 y_1$&\\
				$x_0 + 3h$&	$y_3$	& & $\nabla^2 y_2$&&\\
				&& $\nabla y_3$&&&\\
				$x_0 + 4h$&	$y_4$	& & &&\\ \hline
			\end{tabular}\\
		\end{center}
		Let
		\[ y(x) = a_0 + a_1 (x - x_n) + a_2 (x-x_n)(x-x_{n-1}) + \cdots + a_n (x-x_n)(x-x_{n-1})\ldots (x-x_0) \]
		be a polynomial that satisfies the $n+1$ tabulated data. Then, the expression becomes:
		\begin{dmath}\label{newton-back}			
			y(x) = y_0 + \dfrac{\nabla y_n}{h} (x-x_n) + \dfrac{\nabla^2 y_n}{2! h^2} (x-x_n)(x-x_{n-1})+ \cdots + \dfrac{\nabla^n y_n}{n! h^n} (x-x_n)(x-x_{n-1})\cdots (x-x_0)
		\end{dmath}
		Now, using the substitution $\dfrac{x-x_n}{h} = p$, the equation \ref{newton-back} becomes:
		\begin{equation}\label{newton-back-p}
			\boxed{y(p) = y_n + \nabla y_n p + \dfrac{\nabla^2 y_n}{2!} p(p+1) + \cdots + \dfrac{\nabla^n y_n}{n!} p(p+1)(p+2)\cdots (p+n-1)}
		\end{equation}
		
		\subsection{Lagrange's Interpolation Polynomial}
		
		\begin{dmath*}
			y(x) = a_0 (x-x_1)(x-x_2)\ldots(x-x_n) + a_1 (x-x_0)(x-x_2)\ldots(x-x_n) + a_2 (x-x_0)(x-x_1)(x-x_3)\ldots(x-x_n) + \cdots + a_n (x-x_0)(x-x_1)(x-x_2)\ldots(x-x_{n-1})
		\end{dmath*}
		%
		On simplification:
		\begin{dmath}\label{lagrange}
 			\boxed{y(x) = y_0 \left(\dfrac{ (x-x_1)(x-x_2)\ldots(x-x_n)}{(x_0-x_1)(x_0-x_2)\ldots(x_0-x_n)}\right) + y_1 \left(\dfrac{(x-x_0)(x-x_2)\ldots(x-x_n)}{(x_1-x_0)(x_1-x_2)\ldots(x_1-x_n)} \right) + \cdots + y_n \left(\dfrac{(x-x_0)(x-x_1)\ldots(x-x_{n-1})}{(x_n-x_0)(x_n-x_1)\ldots(x_n-x_{n-1})}\right)}
		\end{dmath}
	
	\subsection{Newton's Divided Difference formula}
	\begin{dmath}\label{newton-divided}
		F[x] = F[x_0] + (x-x_0)F[x_0,x_1] + (x-x_0)(x-x_1)F[x_0,x_1,x_2] + (x-x_0)(x-x_1)(x-x_2)F[x_0,x_1,x_2,x_3]+ \cdots + (x-x_0)(x-x_1)(x-x_2)\cdots(x-x_{n-1})F[x_0,x_1,\ldots,x_n]
	\end{dmath}
	where $F[x_i,x_{i+1}] = \dfrac{F[x_{i+1}] - F[x_i]}{x_{i+1} - x_i}$
	
	\section{Numerical Differentiation}
	\subsection{Using forward difference}
	By using the Newton's forward difference equation given by \ref{newton-for-u},
	\begin{dmath*}
		y(u) = y_0 + \Delta y_0u + \dfrac{\Delta^2 y_0}{2!} u(u-1) + \dfrac{\Delta^3 y_0}{3!} u(u-1)(u-2) + \dfrac{\Delta^4 y_0}{4!} u(u-1)(u-2)(u-3) + \dfrac{\Delta^5 y_0}{5!} u(u-1)(u-2)(u-3)(u-4) \\+ \cdots + \dfrac{\Delta^n y_0}{n!} u(u-1)(u-2)\cdots (u-(n-1))
	\end{dmath*}
	\begin{dmath*}
		y(u) = y_0 + \Delta y_0u + \dfrac{\Delta^2 y_0}{2!} (u^2-u) + \dfrac{\Delta^3 y_0}{3!} (u^3 - 3u^2 + 2u ) + \dfrac{\Delta^4 y_0}{4!} ( u^4 - 6u^3 + 11u^2 -6u) + \dfrac{\Delta^5 y_0}{5!} (u^{5}-10 u^{4}+35 u^{3}-50 u^{2}+24 u)+ \dfrac{\Delta^6 y_0}{6!}(u^{6}-15 u^{5}+85 u^{4}-225 u^{3}+250 u^{2} - 120u) + \cdots + \dfrac{\Delta^n y_0}{n!} u(u-1)(u-2)\cdots (u-(n-1))
	\end{dmath*}
	Now, since $u = \left(\dfrac{x-x_0}{h}\right)$,
	\begin{dmath*}
		\dfrac{dy}{dx} = \dfrac{dy}{du} \cdot \dfrac{du}{dx} = \dfrac{dy}{du} \left(\dfrac{1}{h}\right) = \dfrac{1}{h}\cdot\dfrac{dy}{du}
	\end{dmath*}
	\begin{equation*}
		\therefore \dfrac{dy}{dx} = \dfrac{1}{h}\cdot\dfrac{dy}{du}
	\end{equation*}
	On simplification after differentiating each term of equation \ref{newton-for-u}, 
	\begin{dmath}\label{newton-for-diff}
		\boxed{\dfrac{dy}{dx} = \dfrac{1}{h} \left[ \Delta y_0 + \dfrac{\Delta^2 y_0}{2!} (2u-1) + \dfrac{\Delta^3 y_0}{3!} (3u^2-6u+2) + \dfrac{\Delta^4 y_0}{4!} (4u^3 - 18u^2 + 22u - 6) + \cdots\right]}
	\end{dmath}
	\begin{dmath}\label{newton-for-diff-6}
		\dfrac{dy}{dx} = \dfrac{1}{h} \left[ \Delta y_0 + \dfrac{\Delta^2 y_0}{2!} (2u-1) + \dfrac{\Delta^3 y_0}{3!} (3u^2-6u+2) + \dfrac{\Delta^4 y_0}{4!} (4u^3 - 18u^2 + 22u - 6) + \dfrac{\Delta^5 y_0}{5!} (5u^4 - 40 u^3 + 105 u^2 - 100 u + 24) + \dfrac{\Delta^6 y_0}{6!} (6u^5-75u^4 + 340u^3 - 675 u^2 + 548 u - 120) \right]
	\end{dmath}
	On differentiating again,
	\begin{dmath}\label{newton-for-diff2}
		\boxed{\dfrac{d^2y}{dx^2} = \dfrac{1}{h^2} \left[ \Delta^2 y_0 + \dfrac{\Delta^3 y_0}{3!} (6u-6) + \dfrac{\Delta^4 y_0}{4!} (12u^2 - 36u + 22) + \cdots + \right]}
	\end{dmath}
	\begin{dmath}\label{newton-for-diff2-6}
		\dfrac{d^2y}{dx^2} = \dfrac{1}{h^2} \left[ \Delta^2 y_0 + \dfrac{\Delta^3 y_0}{3!} (6u-6u) + \dfrac{\Delta^4 y_0}{4!} (12u^2 - 36u + 22) + \dfrac{\Delta^5 y_0}{5!} (20u^3 - 120 u^2 + 210 u - 100) + \dfrac{\Delta^6 y_0}{6!} (30u^4-300u^3 + 1020u^2 - 1350 u + 548) \right]
	\end{dmath}
	\subsection{Using backward difference}
	\begin{dmath}\label{newton-back-diff}
		\boxed{\dfrac{dy}{dx} = \dfrac{1}{h} \left[ \nabla y_n + \dfrac{\nabla^2 y_n}{2!} (2p+1) + \dfrac{\nabla^3 y_n}{3!} (3p^2+6p+2) + \dfrac{\nabla^4 y_n}{4!} (4p^3 + 18p^2 + 22p + 6) + \cdots\right]}
	\end{dmath}
	\begin{dmath}\label{newton-back-diff2}
		\boxed{\dfrac{d^2y}{dx^2} = \dfrac{1}{h^2} \left[ \nabla^2 y_n + \dfrac{\nabla^3 y_n}{3!} (6p+6) + \dfrac{\nabla^4 y_n}{4!} (12p^2 + 36p + 22) + \cdots + \right]}
	\end{dmath}
	
	\section{Numerical Integration}
	Using the Newton's forward difference formula given by equation \ref{newton-for-u},
	\[y(u) = y_0 + \Delta y_0u + \dfrac{\Delta^2 y_0}{2!} u(u-1) + \cdots + \dfrac{\Delta^n y_0}{n!} u(u-1)(u-2)\cdots (u-(n-1))\]
	where $u = \dfrac{x-x_0}{h}$. Then, $du = \dfrac{dx}{h}$.
	$\implies dx = h\cdot du$.\\
	Now, $x=x_0\implies u=0,\ x=x_n\implies u = n$.
	\[\therefore \int_{x_0}^{x_n} f(x)dx = h\int_{0}^{n} f(u) du \]
	\subsection{General Quadrature formula}
	On evaluating the above integral using the expression for $f(u)$ given by equation \ref{newton-for-u}, the \textbf{general quadrature formula} is obtained.
	\begin{equation}\label{gen-quad}
		\int_{x_0}^{x_n} f(x)dx = h \left[y_0 n + \Delta y_0 \dfrac{n^2}{2} + \dfrac{\Delta^2 y_0}{2!}\left( \dfrac{n^3}{3} - \dfrac{n^2}{2}\right) + \dfrac{\Delta^3 y_0}{3!}\left( \dfrac{n^4}{4} - n^3 + n^2\right) + \dfrac{\Delta^4 y_0}{4!}\left( \dfrac{n^5}{5} - \dfrac{3n^2}{2} + \dfrac{11n^3}{3} - 3n^2\right) + \cdots\right]
	\end{equation}
	\subsection{Trapezoidal Rule for Numerical Integration}
	Putting $n=1$ in equation \ref{gen-quad} and taking the curve through $(x_0,y_0)$ as a straight line,
	\[ \int_{x_1}^{x_2} f(x)dx = \dfrac{h}{2}\left(y_1 + y_2\right) \]
	\[ \int_{x_{k-1}}^{x_k} f(x)dx = \dfrac{h}{2}\left(y_{k-1} + y_k\right)\ \forall\ k=1,2,\ldots,n\]
	\begin{equation}
		\boxed{\int_{x_0}^{x_n}f(x)dx = \dfrac{h}{2}\left(y_0 + y_n\right) + h \left(y_1 + y_2 + \cdots + y_n\right)}
	\end{equation}
	\subsection{Simpson's one-third Rule}
		Putting $n=2$ in equation \ref{gen-quad}, and taking the curve through $(x_0 , y_0), (x_1, y_1)$ and $(x_2,y_2)$ as a parabola,
		\begin{equation}\label{simpson-13}
			\boxed{\int_{x_0}^{x_n}f(x)dx = \dfrac{h}{3}\left[(y_0 + y_n) + 4(y_1 + y_3 + \ldots + y_{n-1}) + 2(y_2 + y_4 + \ldots + y_{n-2})\right]}
		\end{equation}
	
	\subsection{Simpson's three-eighth Rule}
		Putting $n=2$ in equation \ref{gen-quad}, and taking the curve through $(x_0 , y_0), (x_1, y_1), (x_2,y_2)$ and $(x_3,y_3)$ as a polynomial of third degree,
		\begin{equation}\label{simpson-38}
		\boxed{\int_{x_0}^{x_n}f(x)dx = \dfrac{3h}{8}\left[(y_0 + y_n) + 3(y_1 + y_2 + y_4 + y_5 + y_7 \ldots + y_{n-1}) + 2(y_3 + y_6 + \ldots + y_{n-3})\right]}
		\end{equation}
	\subsection{Boole's Rule}
	\begin{equation}\label{boole}
		\int_{x_0}^{x_4} y\ dx = \dfrac{2h}{45}\left[ 7y_0 + 32y_1 + 32 y_3 + 7y_4 \right]
	\end{equation}
	\subsection{Weddle's Rule}
	\begin{equation}\label{weddle}
		\int_{x_0}^{x_6} y\ dx = \dfrac{3h}{10}\left[y_0 + 5y_1 + y_2 + 6y_3 + y_4 + 5 y_5 + y_6 \right]
	\end{equation}
	
	\section{Numerical Solutions of Algebraic and Transcendental equations}
	
%	\subsection{Bisection Method}
	
	\subsection{Iteration Method}
	For finding the root of an equation of the form $f(x)=0$, the equation should be reduced to the form $x=\phi(x)$ such that $\phi(x)$ is continuously differentiable and $\left|\phi(x)\right|<1$.
	
	\subsection{\textit{Regula-Falsi Method} or Method of false position}
	
	\[x_2 =  x_0 - \dfrac{x_1 - x_0}{f(x_1) - f(x_0)}f(x_0)\]
	
	\subsection{Newton-Ralphson Method}
	
	The Taylor series for a continuous function $f(x)$ is given by
	\begin{equation}\label{taylor}
		f(x) =  f(a) + \dfrac{f'(a)}{1!} (x-a) + \dfrac{f''(a)}{2!}(x-a)^2 + \cdots + \dfrac{f^n (a)}{n!}(x-a)^n + \cdots
	\end{equation}
	For a continuous function which is differentiable infinitely many times, for small values of $h$, (where $h = x_1 - x_0 $) $f(x)$ can be approximated as:	
	\[f(x) = f(x_0) + h\ f'(x_0)\]
	On rearranging, $h = -\dfrac{f(x_0)}{f'(x_0)}$. On further simplification, and eliminating $h$ from the expression, 
	\begin{equation}\label{new-ralph}
		x_n = x_{n-1} - \dfrac{f(x_{n-1})}{f'(x_{n-1})} 
	\end{equation}
	
	\section{Numerical solutions of first order linear differential equations}
	
	\subsection{Picard's method}
	\[  \dydx = f(x,y) \]
	The first approximation is given by
	\begin{equation}\label{Picard}
		y_1 = y_0 + \int_{x_0}^{x} f(x,y_0)\ dx
	\end{equation}
	The second approximation is given by:
	\[ y_2 = y_0 + \int_{x_0}^{x} f(x,y_1)\ dx \]
	Similarly, third approximation is
	\[ y_3 = y_0 + \int_{x_0}^{x} f(x,y_2)\ dx \]
	
	\subsection{Euler-Cauchy method}
	\[ y_1 = y_0 + h\ f(x_0 , y_0) \]
	\[ y_2 = y_1 + h\ f(x_0 + h, y_1)\]
	\[ y_3 = y_2 + h\ f(x_0 + 2h, y_2)\]
	\[ y_4 = y_3 + h\ f(x_0 + 3h, y_3)\]
		\emph{i.e.,}
	\begin{equation}\label{Euler}
	y_n = y_{n-1} + h\ f\left( x_0 + (n-1) h, y_n-1 \right)
	\end{equation}
	
%	\subsection{Modified Euler's method}
	
	\subsection{Runge-Kutta fourth order method}
	\[ \dydx = f(x,y),\quad y(x_0) = y_0 \]
	Calculate successively:
	\begin{eqnarray}
		k_1 &=& hf\left(x_0,y_0\right)\\
		k_2 &=& hf\left(x_0 + \dfrac{1}{2}h,\ y_0 + \dfrac{1}{2}k_1\right)\\
		k_3 &=& hf\left(x_0 + \dfrac{1}{2}h,\ y_0 + \dfrac{1}{2}k_2\right)\\
		k_4 &=& hf\left(x_0 + h, y_0 + k_3\right)
	\end{eqnarray}
	Finally compute:
	\begin{equation}\label{RungeKutta}
		k = \dfrac{1}{6}\left(k_1 + 2k_2 + 2k_3 + k_4\right)
	\end{equation}
	\section*{References}
	Dr. B S Grewal, \textit{Higher Engineering Mathematics}, Khanna Publishers
\end{document}